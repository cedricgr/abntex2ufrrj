%% modelo-trabalho-academico.tex, v-1.0.0 Cedric Graebin
%%
%% Based on the original work by the abnTeX2 group
%% Copyright 2012-2018 by abnTeX2 group at http://www.abntex.net.br/ 
%%
%% This work may be distributed and/or modified under the
%% conditions of the LaTeX Project Public License, either version 1.3
%% of this license or (at your option) any later version.
%% The latest version of this license is in
%%   http://www.latex-project.org/lppl.txt
%% and version 1.3 or later is part of all distributions of LaTeX
%% version 2005/12/01 or later.
%%
%% This work has the LPPL maintenance status `maintained'.
%% 
%% The Current Maintainer of this work is Cedric Graebin.
%% Further information are available on 
%%   http://github.com/cedricgr/abntex2ufrrj
%%
%% This work consists of the files abntex2ufrrj.sty, modelo-trabalho-academico.tex and license.txt
%%
%% WARNING: as part of the UFRRJ's Thesis and Dissertation Style Guide, the University's
%% logomark must be included on the document's cover. This logomark (logomarca.jpg) 
%% is (c) UFRRJ and it is included as a courtesy to avoid compile errors

% ------------------------------------------------------------------------
% ------------------------------------------------------------------------
% abnTeX2: Modelo de Trabalho Academico (tese de doutorado, dissertacao de
% mestrado e trabalhos monograficos em geral) em conformidade com 
% ABNT NBR 14724:2011: Informacao e documentacao - Trabalhos academicos -
% Apresentacao
% ------------------------------------------------------------------------
% ------------------------------------------------------------------------

\documentclass[
	% -- opções da classe memoir --
	12pt,				% tamanho da fonte
	openany,			% capítulos começam em qualquer página (não insere página vazia cf. Manual de Teses)
	twoside,			% Para impressão em frente/verso
	a4paper,			% tamanho do papel. 
	% -- opções da classe abntex2 --
	chapter=TITLE,		% títulos de capítulos convertidos em letras maiúsculas
	%section=TITLE,		% títulos de seções convertidos em letras maiúsculas
	%subsection=TITLE,	% títulos de subseções convertidos em letras maiúsculas
	%subsubsection=TITLE,% títulos de subsubseções convertidos em letras maiúsculas
	% -- opções do pacote babel --
	english,			% idioma adicional para hifenização
	%french,				% idioma adicional para hifenização
	%spanish,			% idioma adicional para hifenização
	brazil				% o último idioma é o principal do documento
	]{abntex2}

% ---
% Pacotes básicos 
% ---
\usepackage{times}				% Usa a fonte Adobe Times PostScript T1 (cf. Manual de Teses)		
\usepackage[T1]{fontenc}		% Selecao de codigos de fonte.
\usepackage[utf8]{inputenc}		% Codificacao do documento (conversão automática dos acentos)
\usepackage{indentfirst}		% Indenta o primeiro parágrafo de cada seção.
\usepackage{color}				% Controle das cores
\usepackage{cmap}				% Torna o texto e conteúdo PDF selecionável e copiável
\usepackage{microtype} 			% para melhorias de justificação


% ---
		
% ---
% Pacotes adicionais, usados apenas no âmbito do Modelo Canônico do abnteX2ufrrj
% ---
\usepackage{lipsum}				% para geração de dummy text. Esta linha pode ser comentada.
% ---

% ---
% Pacotes de citações
% ---
\usepackage[brazilian,hyperpageref]{backref}	 % Paginas com as citações na bibl
\usepackage[alf]{abntex2cite}	% Citações padrão ABNT usando BibTeX

% ---
% Figuras
% ---
\usepackage{graphicx}			% Inclusão de gráficos

% pacotes adicionais como sugestão:
% \usepackage{todonotes} 		% Para comentários no PDF
% \usepackage{pdfx}				% Para tornar o PDF compatível com os padrões PDF/X ou PDF/A. Requer arquivo XMPDATA

% configurações para a UFRRJ
\usepackage{abntex2ufrrj}

% --- 
% CONFIGURAÇÕES DE PACOTES
% --- 

% ---
% Configurações do pacote backref
% Usado sem a opção hyperpageref de backref
\renewcommand{\backrefpagesname}{Citado na(s) página(s):~}
% Texto padrão antes do número das páginas
\renewcommand{\backref}{}
% Define os textos da citação
\renewcommand*{\backrefalt}[4]{
	\ifcase #1 %
		Nenhuma citação no texto.%
	\or
		Citado na página #2.%
	\else
		Citado #1 vezes nas páginas #2.%
	\fi}%
% ---

% ---
% Informações de dados para CAPA e FOLHA DE ROSTO
% ---
\titulo{Insira o título do Trabalho aqui}
\autor{Insira nome do autor}
\local{Seropédica}
\data{07 de Outubro de 2019} 			% altere a data para o dia da defesa
\orientador{Nome do Orientador}
\coorientador{Nome do Co-Orientador} 	%deixe vazio para que não seja impresso
\instituicao{%
  Universidade Federal Rural do Rio de Janeiro - UFRRJ
  \par
  Instituto de Química
  \par
  Programa de Pós-Graduação em Química}
\tipotrabalho{Tese}
% comandos personalizados abntex2-ufrrj
\instituto{Instituto de Química}
\programapg{Curso de Pós-Graduação em Química}

% O preambulo deve conter o tipo do trabalho, o objetivo, 
% o nome da instituição e a área de concentração 
\preambulo{Tese submetida como requisito parcial para a obtenção do grau de Doutor em Química no Curso de Pós-Graduação em Química, Área de Concentração em Química Orgânica}
%\ preambulo{Dissertação submetida como requisito parcial para a obtenção do grau de Mestre em Química no Curso de Pós-Graduação em Química, Área de Concentração em Química Orgânica}
% ---

% ---
% Configurações de aparência do PDF final

% alterando o aspecto da cor azul
\definecolor{blue}{RGB}{41,5,195}

% informações do PDF. Altere o PDFKEYWORDS
\makeatletter
\hypersetup{
     	%pagebackref=true,
		pdftitle={\@title}, 
		pdfauthor={\@author},
    	pdfsubject={\imprimirpreambulo},
	    pdfcreator={LaTeX with abnTeX2},
		pdfkeywords={abnt}{latex}{abntex}{abntex2}{trabalho acadêmico}, 
		colorlinks=true,       		% false: boxed links; true: colored links
    	linkcolor=blue,          	% color of internal links
    	citecolor=blue,        		% color of links to bibliography
    	filecolor=magenta,      		% color of file links
		urlcolor=blue,
		bookmarksdepth=4
}
\makeatother
% --- 

% ---
% compila o indice
% ---
\makeindex
% ---

% ----
% Início do documento
% ----
\begin{document}
    
% Espaçamento de linhas conforme Manual de Teses da UFRRJ
\OnehalfSpacing 	% para versão preliminar (banca)
% \SingleSpacing 	% para versão final p/ biblioteca    

% Seleciona o idioma do documento (conforme pacotes do babel)
%\selectlanguage{english}
\selectlanguage{brazil}

% Retira espaço extra obsoleto entre as frases.
\frenchspacing 

% ----------------------------------------------------------
% ELEMENTOS PRÉ-TEXTUAIS
% ----------------------------------------------------------
\pretextual

% ---
% Capa
% ---
\imprimircapa
% ---

% ---
% Folha de rosto
% (o * indica que haverá a ficha bibliográfica)
% ---

\imprimirfolhaderosto*
% ---

% ---
% Inserir a ficha bibliografica
% ---

% Isto é um exemplo de Ficha Catalográfica, ou ``Dados internacionais de
% catalogação-na-publicação''. Você pode utilizar este modelo como referência. 
% Porém, provavelmente a biblioteca da sua universidade lhe fornecerá um PDF
% com a ficha catalográfica definitiva após a defesa do trabalho. Quando estiver
% com o documento, salve-o como PDF no diretório do seu projeto e substitua todo
% o conteúdo de implementação deste arquivo pelo comando abaixo:
%
% \begin{fichacatalografica}
%     \includepdf{fig_ficha_catalografica.pdf}
% \end{fichacatalografica}

\begin{fichacatalografica}
	\sffamily
	\vspace*{\fill}					% Posição vertical
	\begin{center}					% Minipage Centralizado
	\fbox{\begin{minipage}[c][8cm]{13.5cm}		% Largura
	\small
	\imprimirautor
	%Sobrenome, Nome do autor
	
	\hspace{0.5cm} \imprimirtitulo  / \imprimirautor. --
	\imprimirlocal, \imprimirdata-
	
	\hspace{0.5cm} \thelastpage p. : il. (algumas color.) ; 30 cm.\\
	
	\hspace{0.5cm} \imprimirorientadorRotulo~\imprimirorientador\\
	
	\hspace{0.5cm}
	\parbox[t]{\textwidth}{\imprimirtipotrabalho~--~\imprimirinstituicao,
	\imprimirdata.}\\
	
	\hspace{0.5cm}
		1. Palavra-chave1.
		2. Palavra-chave2.
		2. Palavra-chave3.
		I. Orientador.
		II. Universidade xxx.
		III. Faculdade de xxx.
		IV. Título 			
	\end{minipage}}
	\end{center}
    \vspace*{\fill}	
\end{fichacatalografica}
% ---

% ---
% Inserir errata
% ---
%\begin{errata}
%Elemento opcional da \citeonline[4.2.1.2]{NBR14724:2011}. Exemplo:
%
%\vspace{\onelineskip}
%
%FERRIGNO, C. R. A. \textbf{Tratamento de neoplasias ósseas apendiculares com
%reimplantação de enxerto ósseo autólogo autoclavado associado ao plasma
%rico em plaquetas}: estudo crítico na cirurgia de preservação de membro em
%cães. 2011. 128 f. Tese (Livre-Docência) - Faculdade de Medicina Veterinária e
%Zootecnia, Universidade de São Paulo, São Paulo, 2011.
%
%\begin{table}[htb]
%\center
%\footnotesize
%\begin{tabular}{|p{1.4cm}|p{1cm}|p{3cm}|p{3cm}|}
%  \hline
%   \textbf{Folha} & \textbf{Linha}  & \textbf{Onde se lê}  & \textbf{Leia-se}  \\
%    \hline
%    1 & 10 & auto-conclavo & autoconclavo\\
%   \hline
%\end{tabular}
%\end{table}
%
%\end{errata}
% ---

% ---
% Inserir folha de aprovação
% ---

% Isto é um exemplo de Folha de aprovação, elemento obrigatório da NBR
% 14724/2011 (seção 4.2.1.3). Você pode utilizar este modelo até a aprovação
% do trabalho. Após isso, substitua todo o conteúdo deste arquivo por uma
% imagem da página assinada pela banca com o comando abaixo:
%
% \begin{folhadeaprovacao}
% \includepdf{folhadeaprovacao_final.pdf}
% \end{folhadeaprovacao}
%
\begin{folhadeaprovacao}
    % --
    % A folha de aprovação foi adequada ao Manual de Teses da UFRRJ
    % --
    
    \noindent \textbf{UNIVERSIDADE FEDERAL RURAL DO RIO DE JANEIRO \\
    \MakeTextUppercase{\imprimirinstituto} \\
    \MakeTextUppercase{\imprimirprogramapg}}

    \vfill
      
    \begin{center}
        \textbf{\MakeTextUppercase{\imprimirautor}}
    \end{center}

    \vfill
    
    \noindent \imprimirpreambulo
       
    \noindent \imprimirtipotrabalho{} aprovada em 24 de novembro de 2012.

    % --
    % Retire os comentários para acrescentar mais assinaturas abaixo. 
    % Segundo o Manual de Teses da UFRRJ, a ordem é
    % Orientador, Membros Externos e Internos Titulares, Membros Suplentes
    %--
    % Nome Completo. Título (Dr. ou Ph.D.). Sigla da Instituição de Origem. 
    % Abaixo, entre parênteses, (Orientador), (Membro Externo) ou (Membro Interno)
    %-- 
    
    \assinatura{\imprimirorientador \\ (Orientador)} 
    \assinatura{Professor \\ (Convidado 1)}
    \assinatura{Professor \\ (Convidado 2)}
    %\assinatura{Professor \\ (Convidado 3)}
    %\assinatura{Professor \\ (Convidado 4)}
    
    \vfill
        
\end{folhadeaprovacao}
% ---

% ---
% Dedicatória
% ---
\begin{dedicatoria}
   \vspace*{\fill}
   \centering
   \noindent
   \textit{ Insira texto da dedicatória aqui.} \vspace*{\fill}
\end{dedicatoria}
% ---

% ---
% Agradecimentos
% ---
\begin{agradecimentos}
Insira os agradecimentos aqui.

\end{agradecimentos}
% ---

% ---
% Epígrafe
% ---
%\begin{epigrafe}
%    \vspace*{\fill}
%	\begin{flushright}
%		\textit{``Não vos amoldeis às estruturas deste mundo, \\
%		mas transformai-vos pela renovação da mente, \\
%		a fim de distinguir qual é a vontade de Deus: \\
%		o que é bom, o que Lhe é agradável, o que é perfeito.\\
%		(Bíblia Sagrada, Romanos 12, 2)}
%	\end{flushright}
%\end{epigrafe}
% ---

% ---
% RESUMOS
% ---

% resumo em português
\setlength{\absparsep}{18pt} % ajusta o espaçamento dos parágrafos do resumo
\begin{resumo}
% altere conforme necessidade, especialmente ano e número de páginas    
SOBRENOME, Nomes do Autor. \textbf{\imprimirtitulo}.
2000. 128p \imprimirtipotrabalho{} (Mestrado em Agronomia, Ciência do Solo). \imprimirinstituto, Departamento de X, Universidade Federal Rural do Rio de Janeiro, Seropédica, RJ, 2000.

% O resumo propriamente dito vai aqui.
Insira o texto do resumo aqui.

\textbf{Palavras-chave}: latex. abntex. editoração de texto.
\end{resumo}

% resumo em inglês
\begin{resumo}[Abstract]
 \begin{otherlanguage*}{english}
   SURNAME, Author Name. \textbf{Title (in english)}. 2000. 128p. Thesis (Graduate Program). \imprimirinstituto, Departamento de X, Universidade Federal Rural do Rio de Janeiro, Seropédica, RJ, 2000.
   
   Insert English Abstract here.

   \vspace{\onelineskip}
 
   \noindent 
   \textbf{Keywords}: latex. abntex. text editoration.
 \end{otherlanguage*}
\end{resumo}

% resumo em francês 
%\begin{resumo}[Résumé]
% \begin{otherlanguage*}{french}
%    Il s'agit d'un résumé en français.
% 
%   \textbf{Mots-clés}: latex. abntex. publication de textes.
% \end{otherlanguage*}
%\end{resumo}

% resumo em espanhol
%\begin{resumo}[Resumen]
% \begin{otherlanguage*}{spanish}
%   Este es el resumen en español.
%  
%   \textbf{Palabras clave}: latex. abntex. publicación de textos.
% \end{otherlanguage*}
%\end{resumo}
% ---

% ---
% inserir lista de ilustrações
% ---
\pdfbookmark[0]{\listfigurename}{lof}
\listoffigures*
\cleardoublepage
% ---

% ---
% inserir lista de esquemas
% ---
\pdfbookmark[0]{\listofesquemasname}{loe}
\listofesquemas*
\cleardoublepage
% ---

% ---
% inserir lista de quadros. Retire os comentários deste trecho caso use quadros em seu trabalho acadêmico.
% ---
%\pdfbookmark[0]{\listofquadrosname}{loq}
%\listofquadros*
%\cleardoublepage
% ---

% ---
% inserir lista de tabelas
% ---
\pdfbookmark[0]{\listtablename}{lot}
\listoftables*
\cleardoublepage
% ---

% ---
% inserir lista de abreviaturas e siglas
% ---
\begin{siglas}
  \item[ABNT] Associação Brasileira de Normas Técnicas
\end{siglas}
% ---

% ---
% inserir lista de símbolos
% ---
%\begin{simbolos}
%  \item[$ \Gamma $] Letra grega Gama
%  \item[$ \Lambda $] Lambda
%  \item[$ \zeta $] Letra grega minúscula zeta
%  \item[$ \in $] Pertence
%\end{simbolos}
% ---

% ---
% inserir o sumario
% ---
\pdfbookmark[0]{\contentsname}{toc}
\tableofcontents*
\cleardoublepage
% ---



% ----------------------------------------------------------
% ELEMENTOS TEXTUAIS
% ----------------------------------------------------------
\textual
\pagestyle{plain}
\setcounter{page}{1} % de acordo com o Manual de Teses, a primeira página da Introdução é a página 1

% ----------------------------------------------------------
% Introdução (exemplo de capítulo sem numeração, mas presente no Sumário)
% ----------------------------------------------------------
\chapter{Introdução}


% ----------------------------------------------------------



% ---
% Capitulo com exemplos de comandos inseridos de arquivo externo 
% ---
%\include{abntex2-modelo-include-comandos}
% ---

\chapter{Revisão da Literatura}\label{cap_trabalho_academico}

%\begin{figure}
%    \includegraphics[width=3cm]{logomarca.jpg}
%    \caption{Uma figura} \label{fig:logomarca-ufrrj}
%\end{figure}


\chapter{Objetivos}

\chapter{Material e Métodos}

\chapter{Resultados e Discussão}

% ----------------------------------------------------------
% Finaliza a parte no bookmark do PDF
% para que se inicie o bookmark na raiz
% e adiciona espaço de parte no Sumário
% ----------------------------------------------------------
\phantompart

% ---
% Conclusão
% ---
\chapter{Conclusão}
% ---


% ----------------------------------------------------------
% ELEMENTOS PÓS-TEXTUAIS
% ----------------------------------------------------------
\postextual
% ----------------------------------------------------------

% ----------------------------------------------------------
% Referências bibliográficas. Insira o arquivo .bib aqui
% ----------------------------------------------------------
%\bibliography{abntex2-modelo-references}

% ----------------------------------------------------------
% Glossário
% ----------------------------------------------------------
%
% Consulte o manual da classe abntex2 para orientações sobre o glossário.
%
%\glossary

% ----------------------------------------------------------
% Apêndices
% ----------------------------------------------------------


% ----------------------------------------------------------
% Anexos
% ----------------------------------------------------------

% ---
% Inicia os anexos
% ---
\begin{anexosenv}

% Imprime uma página indicando o início dos anexos
\partanexos

% ---
\chapter{Anexo 1}
% ---

\end{anexosenv}

%---------------------------------------------------------------------
% INDICE REMISSIVO
%---------------------------------------------------------------------
%\phantompart
%\printindex
%---------------------------------------------------------------------

\end{document}
